
In practice researchers also need to choose the degree of the local
polynomials, which additionally reflects an assumption about how smooth the
underlying process is. Specifically, for a first-degree LPR, the process should
not exhibit any corners, discontinuities, or vertical sections. This ensures
that the processes rate of change (i.e., derivative), approximated by the
first-order polynomial term, is well behaved. Higher-order local polynomials
require this smoothness for increasingly complex rates of change. For instance,
a second-degree LPR requires that the rate of change itself is smooth, in turn
ensuring that its rate of change is well behaved. This property should hold for
all $p$ rates of change of a process when using an LPR with $p$ degrees (under
Taylor's theorem). Typically, the degree of the local polynomials is chosen to
be low and odd. This choice reflects a bias-variance tradeoff, where
higher-order polynomials reduce bias but increase variance only when
transitioning from an odd to an even power
\parencite{ruppert_multivariate_1994}.