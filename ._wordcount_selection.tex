In the following, the three analysis methods previously introduced were applied
to depression data from the Leuven clinical study. This study used experience
sampling measures to study the dynamics of anhedonia in individuals with major
depressive disorder \parencite{heininga_dynamical_2019}. This study was
selected for its heterogeneous sample, which includes participants with major
depressive disorder, borderline personality disorder, and healthy controls.
This diversity increases the likelihood of the data exhibiting a range of
(possibly non-linear) dynamics and processes. Specifically,
\textcite{houben_relation_2015} found in their meta-analysis that individuals
with lower psychological well-being tend to experience greater emotional
variability, less emotional stability, and higher emotional inertia. Although,
this finding did not replicate in an analysis of positive affect within the
Leuven clinical study \parencite{heininga_dynamical_2019}. Further, emotional
inertia, the extent to which an emotional state carries over across time
points, has been shown to vary within individuals over time (citation), which
makes it likely that the processes underlying this data are non-stationary.