% -----------------------------------------------------------------------------%
% Title:                                                                       %
% Author: Jan Ian Failenschmid                                                 %
% Created Date: 07-12-2023                                                     %
% -----                                                                        %
% Last Modified: 27-03-2024                                                    %
% Modified By: Jan Ian Failenschmid                                            %
% -----                                                                        %
% Copyright (c) 2024 by Jan Ian Failenschmid                                   %
% E-mail: J.I.Failenschmid@tilburguniveristy.edu                               %
% -----                                                                        %
% License: GNU General Public License v3.0 or later                            %
% License URL: https://www.gnu.org/licenses/gpl-3.0-standalone.html            %
% -----------------------------------------------------------------------------%

\subsection{Generalized Additive Models}

Generalized additive models (GAM) choose a different approach to approximating
the global
non-linear state function, relying on smoothing splines
\parencite{wood_generalized_2006, wood_inference_2020,
    hastie_generalized_1999}.
Spline methods approximate a complex function piecewise by joining
simpler functions, usually low-order polynomials, together at so-called knots.
For this,
it is important to constrain the basis functions to fit together smoothly and
to choose
the knot locations optimally \parencite{tsay_nonlinear_2019}. To circumvent
this, smoothing
splines approximate the complex function as a weighted sum of a collection of
basis functions
that range over the entire domain of the underlying function. In order to do
this, all basis functions are evaluated at each
time point, and the optimal weights for each basis function are found using
penalized regression
methods. The penalty term here controls the smoothness of the approximation and
prevents
over-fitting \parencite{gu_smoothing_2013, wahba_spline_1980}. Subsequently,
the weight of the penalty term
is efficiently optimized using generalized cross-validation
\parencite{wood_generalized_2006,
    golub_generalized_1997}. Common sets of basis functions are natural cubic
splines
\parencite{tsay_nonlinear_2019} or thin plate splines
\parencite{wood_thin_2003}.

GAMs build upon smoothing splines and combine them in the form of a generalized
linear model.
Meaning that the parameters of the general linear model smooth functions
instead
\parencite{wood_generalized_2006, hastie_generalized_1999}. This makes it
possible to formulate time-
varying auto-regressive models within the GAM framework
\parencite{bringmann_changing_2017,
    bringmann_modeling_2015}. In these models, any of the terms, intercept, or
auto-regressive,
is modeled as varying smoothly over time, simultaneously describing non-linear
trends and changes in the auto-regressive effect.

GAMs rely on the assumption that the underlying state function is smooth.
One example of a process that may satisfy this assumption is emotional valence
\parencite{bringmann_modeling_2015}. Then, GAMs provide information about
the underlying process, in the form of interpretable time-varying intercept and
autocorrelation
parameters. However, the interpretation of these parameters is complicated by
the interaction
between trend and autocorrelation effects in non-stationary models
\parencite{bauldry_nonlinear_2018, bollen_autoregressive_2004}.
Additionally, it is possible to interpret the coefficients for the basis
functions,
as the degree to which these basic trends are evident in the smooth functions
describing the
time-varying parameters. However, this interpretation is again complicated by
the presence
of many other terms in the model and the penalization.
