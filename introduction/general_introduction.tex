% Broad introduction
Psychological constructs are increasingly understood as being part of dynamic 
systems \parencite{nesselroade_studying_2004, wang_investigating_2012, 
gelfand_dynamical_2012}. As such, they naturally vary not only between but also 
within individuals. Thus, researchers are becoming more aware of the importance of 
studying how these constructs change over time to gain valuable insights into the 
underlying processes \parencite{molenaar_manifesto_2004}. This has led to a sharp 
increase in experience sampling (ESM) studies, in which 
individuals complete short questionnaires assessing psychological constructs over 
a longer period of time \parencite{scollon_experience_2003, 
miller_smartphone_2012}. The resulting 
intensive longitudinal data (ILD) is then used to study how psychological 
constructs change over time within individuals in their natural context and how 
these dynamics vary between individuals. 

% Motivation
Many of the phenomena that are interesting to study using ESM follow 
non-linear trajectories through time. Typical examples of this are 
second language acquisition \parencite{hohenberger_language_2009, 
murakami_modeling_2016, reid_bifurcations_2019}, the onset 
\parencite{nelson_moving_2017, van_de_leemput_critical_2014}, and treatment of 
psychopathologies \parencite{hayes_change_2007, hosenfeld_major_2015, 
schiepek_complexity_2009}, attitude change \parencite{van_der_maas_sudden_2003}, 
mood instability in bipolar disorder \parencite{bonsall_nonlinear_2012}, and flow
\parencite{ceja_suddenly_2012, ceja_dynamics_2009}. 
Various methods for modeling non-linearity in time series exist, covering the entire spectrum 
from completely data-driven exploratory tools to confirmatory modeling and 
theory-testing frameworks. However, research extensively examining and comparing 
these methods in the context of ILD is scarce. Since many of these methods can 
capture different types of non-linearity, rely on different sets of assumptions, 
and allow researchers to draw distinct kinds of inference, it is oftentimes 
unclear which method is most suited in a specific situation. Because of this, in 
the following, we will review common non-linear time series methods, discuss the 
specific types of non-linearity that are described by each method, and show how 
these methods may be applied to ILD. 

% Working definitions and setting
In ILD, researchers are usually measuring at least one construct over time, using 
imperfect indicators, typically in the form of questionnaires. Hence, it is 
assumed that the observations, which are the specific values that the indicators 
take at a given time point, are caused by the underlying 
state or values of the construct \parencite{mcneish_measurement_2021, 
vogelsmeier_assessing_2023}. The difference between the observations and the 
state constitutes the measurement error. Further, it is important to note that 
this error is qualitatively distinct from dynamic error, which relates to 
disturbances or innovations to the state and carries forward into subsequent time 
points \parencite{schuurman_incorporating_2015}. 

% Give some structure to the following review
When dealing with such data, researchers typically aim to infer the 
system's state from the observations and examine the process of how 
the state evolves over time. This process can be described using a global 
model that establishes a relationship between time and the state values or 
a local model that links the extent of the state's change to its present value. 
Local or dynamic models take the form of difference equations in discrete time 
\parencite{durbin_time_2012, galor_discrete_2007}
and differential equations in continuous time \parencite{van_montfort_continuous_2018}. 
Lastly, non-parametric methods can be used to approximate these models in a data-driven way,
whereas semi- or fully parametric models make it possible to test specific theories about the data.
