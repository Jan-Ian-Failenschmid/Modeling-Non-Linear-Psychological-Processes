% -----------------------------------------------------------------------------%
% Title:                                                                       %
% Author: Jan Ian Failenschmid                                                 %
% Created Date: 07-12-2023                                                     %
% -----                                                                        %
% Last Modified: 27-03-2024                                                    %
% Modified By: Jan Ian Failenschmid                                            %
% -----                                                                        %
% Copyright (c) 2024 by Jan Ian Failenschmid                                   %
% E-mail: J.I.Failenschmid@tilburguniveristy.edu                               %
% -----                                                                        %
% License: GNU General Public License v3.0 or later                            %
% License URL: https://www.gnu.org/licenses/gpl-3.0-standalone.html            %
% -----------------------------------------------------------------------------%

\subsection{Gaussian Process Regression}

Gaussian process (GP) regression follows yet another approach.
Here, a probability distribution is directly
imposed on the set of all possible state functions. This is done by imposing
a Bayesian GP prior on the data. The behavior of this
prior is then defined by a continuous mean and covariance function
\parencite{rasmussen_gaussian_2006, betancourt_robust_2020,
    roberts_gaussian_2013}. To develop an intuition for this, it makes sense to
imagine this probability distribution pointwise, such that for any set of
specific time points,
the functions described by the GP take values, which are in turn described by a
multivariate normal distribution.
The mean and covariance of this multivariate normal distribution are
respectively defined by the mean
and covariance function of the GP. Since this relation holds for any arbitrary
set of time points,
the probability distribution is defined for functions over all time points
\parencite{rasmussen_gaussian_2006,
    betancourt_robust_2020, roberts_gaussian_2013}.

For practical purposes, the mean function is oftentimes defined to be zero and
the
covariance function is chosen to be stationary, such that the covariance
between
two time points only depends on their distance. Nevertheless, these conventions

can be broken, and it is possible to include a specific mean function to model
for example polynomial trends \parencite{ohagan_curve_1978, hwang_how_2023,
    blight_bayesian_1975}. It is also possible to combine multiple covariance
functions or design custom covariance and mean functions
to accommodate desired types of non-linear behavior within the
Gaussian process. Typical choices of covariance functions usually result
in smooth approximations of constant wigglyness, making them most appropriate
for these types of processes.
Covariance functions are typically selected through domain
knowledge or model selection \parencite{richardson_gaussian_2017,
    abdessalem_automatic_2017}. Subsequently, the model is estimated using a
Bayesian
approach by defining hyper-priors for the parameters of the mean and the
covariance function.