\subsection{Research Questions}

While all of these methods are well established in the time series literature,
the differences between time series data and ILD make it unclear to what degree 
these methods are applicable to the latter and which method is best suited in a given 
situation. Importantly, ILD is often composed of time series data from multiple individuals, 
creating a multilevel structure, with several time points collected per individual.
Additionally, observations may be spaced unevenly across time and include time-varying 
and invariant confounders. Lastly, when choosing a method to analyze ILD, researchers 
also need to take into account the available a-priori theory and what type of inference 
is desired. To clarify this, we will investigate the following:

\begin{itemize}
    \item To what degree can each of these methods accommodate different types of non-linearity that can be suspected in ILD?
    \item How well do each of the methods perform at sample sizes common in ILD, when specified correctly or misspecified?
    \item To what degree can each method accommodate hierarchical data structures, unevenly spaced time points, and time-invariant confounders?
    \item What degree of theory is required by each method and what type of inference does it yield? 
\end{itemize}

To elucidate these issues, we will first compare and evaluate the described methods on
simulated ILD. Afterwards, we will show how these methods can be used to gain insight 
into non-linear processes in real ILD.
