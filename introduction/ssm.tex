% -----------------------------------------------------------------------------%
% Title:                                                                       %
% Author: Jan Ian Failenschmid                                                 %
% Created Date: 07-12-2023                                                     %
% -----                                                                        %
% Last Modified: 27-03-2024                                                    %
% Modified By: Jan Ian Failenschmid                                            %
% -----                                                                        %
% Copyright (c) 2024 by Jan Ian Failenschmid                                   %
% E-mail: J.I.Failenschmid@tilburguniveristy.edu                               %
% -----                                                                        %
% License: GNU General Public License v3.0 or later                            %
% License URL: https://www.gnu.org/licenses/gpl-3.0-standalone.html            %
% -----------------------------------------------------------------------------%

\subsection{State Space Models}

When constructing a parametric model for time-series data, one is faced with
the
decision to model the state function globally, locally, or mixed. Global
models,
that try to capture the entire behavior of a non-linear state function, are
oftentimes too simplistic and generalize poorly across time and individuals
\parencite{hunter_two_2022}. Because of this, in many situations, it is easier
to
define a model in terms of a local dynamic. State space models (SSM) are a
flexible modeling framework that combines these local dynamic models with
measurement models, making them suitable for modeling psychological data
\parencite{durbin_time_2012}. While SSMs formally only include lag one
relationships,
it is possible to include higher order lags by extending the
state space \parencite{hunter_state_2018}. Further, it is possible
to redefine the model in continuous time using differential equations. This
approach is
especially beneficial when observations are unequally spaced in time
\parencite{van_montfort_continuous_2018}.

\subsubsection{Latent Change Score Models}

Importantly, when all the dynamic equations in an SSM are linear, this modeling
approach is
very similar to estimating structural equation models with lagged variables
\parencite{asparouhov_dynamic_2018, usami_unified_2019}. One such model is the
latent change score
model (LCS), which models the amount by which the state changes as an
additional latent variable.
This latent change variable is then linearly predicted from the state value at
the previous time
point \parencite{cancer_dynamical_2021, cancer_effectiveness_2023}.
Importantly, this model
highlights how a local linear dynamic results in a global exponential
trend. To describe this exponential trend meaningfully, the model can be
parameterized in terms of an asymptote and a
growth rate. Such that, if the growth rate is smaller than one, the state
converges
to the asymptote and if the growth rate is larger than one, it diverges. One
example of a process
that could follow such a trajectory is the developmental trajectory of
intelligence,
where one might suspect large increases in early childhood
that level off towards an asymptote \parencite{savi_evolving_2021}.

\subsubsection{Regime Switching Models}

Another class of commonly used SSMs in psychology are regime switching models
\parencite{hamaker_regime-switching_2010}.
Here one assumes that the system that underlies the time series has distinct
regimes and that each
regime is governed by a different local dynamic. To accommodate this, regime
switching models
combine multiple local models with a model for how the system switches between
the regimes.
In this way, regime switching models describe both the behavior within each
regime as well as the regime switching simultaneously.
The dynamic equations are usually chosen as part of the ARIMA model family
\parencite{box_time_1970}, such that only
the parameter values of the dynamic equations change between the regimes. The
regime switching is commonly modeled using
one of two methods. First, threshold autoregressive models estimate a threshold
value for one of the
observed variables to model the regime switching
\parencite{tong_threshold_1980}.
Second, Markov switching autoregressive models use a discrete latent variable
that follows a Markov chain
to capture the regime switching \parencite{hamilton_new_1989}. One example of a
process,
in which one might reasonably suspect distinct regimes is flow, where
individuals appear to
suddenly change in and out of a flow state \parencite{ceja_suddenly_2012}.
