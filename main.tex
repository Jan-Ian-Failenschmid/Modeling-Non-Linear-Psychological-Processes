% -----------------------------------------------------------------------------%
% Title:
% % Author: Jan Ian Failenschmid % Created Date: 13-03-2024
% %
% -----                                                                        %
% Last Modified: 20-09-2024                                                    %
% % Modified By: Jan Ian Failenschmid
% %
% %
% %
% -----                                                                        %
% Copyright (c) 2024 by Jan Ian Failenschmid
% % E-mail: J.I.Failenschmid@tilburguniveristy.edu
% %
% -----                                                                        %
% License: GNU General Public License v3.0 or later
% % License URL: https://www.gnu.org/licenses/gpl-3.0-standalone.html
% %
% -----------------------------------------------------------------------------%

\documentclass[man, floatsintext]{apa7}

% Dependencies
\usepackage{csquotes, amsmath, amssymb, mathptmx, enumitem, array,
  enumitem}\usepackage{tabularx, makecell}
\usepackage[american]{babel}
\usepackage[figuresleft]{rotating}
\usepackage[style=apa,sortcites=true,sorting=nyt,backend=biber]{biblatex}
\DeclareLanguageMapping{american}{american-apa} \graphicspath{ {./figures} }
% Gaphics Path
\addbibresource{bibliography.bib} % Literature bibliography
\addbibresource{R_bib.bib} % R bibliography
\DeclareMathOperator*{\argmin}{argmin}
\setcounter{secnumdepth}{3}
\parindent=0pt
\newcolumntype{s}{>{\hsize=.6\hsize}X}
\renewcommand\theadfont{\normalsize\bfseries}
\usepackage{etoolbox}
\AtBeginEnvironment{tabularx}{\setlist[itemize, 1]{wide,
    leftmargin=*, itemsep=0pt, before=\vspace{-\dimexpr\baselineskip +2
      \partopsep}, after=\vspace{-\baselineskip}}}
% Title Page
\title{Modeling Non-Linear Psychological Processes: Reviewing and Evaluating
  Non-parametric Approaches and Their Applicability to Intensive
  Longitudinal Data}

\shorttitle{Modelling Non-Linear Psychological Processes}

\leftheader{Failenschmid}

\authorsnames{{Jan I. Failenschmid}, {Leonie V.D.E Vogelsmeier}, {Joris
      Mulder}, {Joran Jongerling}}

\authorsaffiliations{{Tilburg University}}

\abstract{Here could your abstract be!}

\keywords{Here could your keywords be!}

\authornote{ \addORCIDlink{Jan I. Failenschmid}{0009-0007-5106-7263}

  Correspondence concerning this article should be addressed to Jan I.
  Failenschmid, Tilburg School of Social and Behavioral Sciences: Department of
  Methodology and Statistics, Tilburg University, Warandelaan 2, 5037AB
  Tilburg, Netherlands. E-mail: J.I.Failenschmid@tilburguniversity.edu}

\begin{document}

\maketitle

Psychological constructs are increasingly understood as components of complex
dynamic systems \parencite{nesselroade_studying_2004, wang_investigating_2012}.
This perspective emphasizes that these constructs fluctuate over time and
within individuals. To study these variations and the underlying processes,
researchers are increasingly collecting intensive longitudinal data (ILD) using
ecological momentary assessment (EMA), experience sampling, or similar methods
\parencite{fritz_so_2023}. In these studies, one or more individuals are
assessed at a high frequency (multiple times per day) using brief
questionnaires or passive measurement devices. These rich data allow
researchers to examine complex temporal variations in the underlying
psychological variables within an ecologically valid context and to explain
them through (between-person differences) in within-person processes.

Due to these ILD studies, many non-linear psychological phenomena and processes
have been discovered in recent years. Clear examples of this are the
learning and growth curves observed in intellectual and cognitive development
\parencite{kunnen_dynamic_2012, mcardle_comparative_2002}. In these cases, an
individual's latent ability increases over time, following an intricate
non-linear trajectory from a (person-specific) starting point towards a
(person-specific) asymptote, which reflects the individual's maximum ability.
Additional examples of asymptotic growth over shorter time spans that are
typically studied with ILD include motor skill development
\parencite{newell_time_2001} and second language acquisition
\parencite{de_bot_dynamic_2007}. Figure~\ref{fig:examplar_npn} shows
common model choices for these kinds of processes in the form of an
exponential growth function (a) and a logistic growth function (b).

\begin{figure}[!t]
  \caption{Examples of non-linear processes demonstrated to occur in
    psychological time series}
  \fitfigure{exemplar_no_process_noise.png}
  \figurenote{This figure shows four demonstrated psychological non-linear
    processes. Panels (a) and (b) show exponential and logistic growth curves,
    respectively. Panel (c) shows a cusp catastrophe model. Lastly, panel (d)
    shows a damped oscillator.}
  \label{fig:examplar_npn}
\end{figure}

Another common non-linear phenomenon is switching between seamingly distinct
states that diffre, for instance, in their means. This occurs, for example,
during the sudden perception of cognitive flow, where individuals abruptly
switch from a "normal" state to a flow state and back
\parencite{ceja_suddenly_2012}. Another example is alcohol use relapse, where
patients suddenly switch from an abstinent state to a relapsed state
\parencite{witkiewitz_modeling_2007}. This sudden switching behavior has been
modelled using a cusp catastrophe model. This dynamic model, drawn from
catastrophe theory, naturally leads to mean level switches when varying one of
its parameters \parencite{van_der_maas_sudden_2003,chow_cusp_2015} and has been
exemplified in Figure~\ref{fig:examplar_npn}~(c).

As a final example, one may consider (self-) regulatory systems, which maintain
a desired state by counteracting external perturbations. In these systems, the
degree of regulation often depends on the distance
between the current and the desired states. The common autoregressive model
describes such a system in which the regulation strength depends linearly on
this distance. However, this relationship may also be non-linear, such that the
degree of regulation changes disproportionately with larger distances. Such a
(self-) regulatory model has been used to model, for example, emotion
regulation \parencite{chow_emotion_2005} using a damped oscillator model. This
model is exemplified in Figure~\ref{fig:examplar_npn}~(d).

Although initial evidence for non-linearity in psychological research exists,
theories about the nature and form of non-linear psychological processes remain
scarce \parencite{tan_time-varying_2011}. Frequently, psychological theories
are too general to result in specific hypotheses
\parencite{oberauer_addressing_2019}, such as, for example, about the form of
specific non-linear dynamics. ILD studies could potentially help to refine
these theories through a nuanced understanding of how the involved
psychological variables interact over time. Such refined theories could, for
instance, take the form of dynamic models, such as differential equation
\parencite{cooper_dynamical_2012} or state-space models
\parencite{durbin_time_2012}, that describe how a given process changes over
time. However, in order to develop these types of theories it is first
necessary to identify empirical phenomena in ILD, which can be thought of
"stable and general features of the world that scientists seek to explain"
\parencite{borsboom_theory_2021}. Examples of such phenomena could be the
described state switching or regulatory oscillations, if they generalize beyond
individual data sets and contexts. Formal theories about the underlying process
should then be able to explain these phenomena and different candidate theories
can be compared on their success to do so \parencite{borsboom_theory_2021}.
While the study of non-linear phenomena in ILD is receiving increasingly more
attention in psychology and different statistical techniques are developed to
explore these phenomena \parencite{cui_unlocking_2023,humberg_estimating_2024},
researchers are currently still limited in their ability to infer non-linear
phenomena from ILD\@. One reason for this is a lack of advanced statistical
methods that are flexible enough to adequately capture and explore these
processes, which hinders the development and evaluation of guiding theories.

Due to the absence of adequate statistical methods, non-linear trends in
psychology are often addressed through polynomial or piecewise spline
regression. Polynomial regression \parencite{jebb_time_2015} uses higher-order
terms (e.g., squared or cubed time) as predictors in a standard multiple linear
regression model. While effective for relatively simple non-linear
relationships, particularly those that can be represented as polynomials, this
method has significant limitations and likely leads to invalid results when
applied to more complex processes, such as mean switching or (self-) regulatory
systems (e.g., Figure~\ref{fig:examplar_npn}~c~\&~d). In these cases,
polynomial approximations require many higher-order terms to capture the
process's complex trajectory, which raises the problem of over- or underfitting
the data, causes model instability, and leads to nonsensical inferences (e.g.,
interpolating scores outside the scale range;
\textcite{boyd_divergence_2009,harrell_general_2001,jianan_case_2023}).

An alternative approach is piecewise spline regression, which constructs a
complex non-linear trend by joining multiple simple piecewise functions at
specific points, called knots (e.g., combining multiple cubic functions into a
growth curve with plateaus; \textcite{tsay_nonlinear_2019}). However, spline
regression requires a careful, manual selection of the optimal piecewise
functions and knot locations. This can be problematic in practice because, as
mentioned, precise guiding theories about the functional form of most
psychological processes are lacking \parencite{tan_time-varying_2011}. This
absence of clear guidance can quickly lead to misspecified models and invalid
results.

These limitations in the currently available methods underscore the need for
more sophisticated statistical methods to study and explore non-linear
processes. Various such advanced statistical methods like kernel regression,
Gaussian processes, and smoothing splines are available outside of psychology.
However, these methods have rarely been applied in psychology because they have
not been reviewed for an applied audience, nor have their assumptions and
inference possibilities been evaluated in the context of ILD.\@ As a result, it
is difficult for psychological researchers to select the most suitable method
for a specific context. This challenge is further complicated by the fact that
the ideal statistical method may depend on the characteristics of the
underlying non-linear process, which are generally unknown. Especially, since
the ideal smooth processes, as depicted in Figure~\ref{fig:examplar_npn}, for
which many of these methods were originally developed, are unlikely to occur in
psychological research. This is because psychological constructs are
(especially in EMA research) measured in an environment in which external
influences affect and perturb the construct. These constant perturbations can
quickly result in the non-smooth or rough processes that are typically seen
in psychological time series. It is for instance easy to imagine that one's
happyness may not decrease smooth and gradually, after stubbing one's toe.

To address this important gap, this article reviews three advanced non-linear
analysis methods and evaluates their applicability to typical ILD scenarios
(Section~\ref{method_introduction}). The methods reviewed in this article are
different semi- and non-parametric regression techniques, available in the
open-source software R \parencite{R-base}, which are able to infer non-linear
functions from data while accommodating varying degrees of prior knowledge. In
a simulation study, we compare how well each method recovers different
non-linear processes under common ILD conditions (Section~\ref{simulation}).
Lastly, we demonstrate how the best-performing method can be applied to analyze
an existing dataset (Section~\ref{empirical_example}). Further, to introduce
these methods accessibly and apply them under conditions where software
implementations are available, this article focuses on the univariate
single-subject design.

\section{Non-linear analysis methods}\label{method_introduction}

In the following paragraphs the three selected semi- and non-parametric
regression techniques will be introduced.

\subsection{Local polynomial regression}

The first technique is called local polynomial regression (LPR). Similarly to
regular polynomial regression, LPR approximates the process using polynomial
basis functions (e.g., squared or cubed time). However, instead of using one
large polynomial function to approximate the entire process, LPR estimates
smaller, local polynomials at any point in time. These local polynomials are
then combined into a single non-linear function over the entire set of
observations \parencite{fan_adaptive_1995, ruppert_multivariate_1994,
  fan_local_2018}.

To determine the value that the LPR predicts at a specific time point, the data
is first centered around that point (by shifting the data along the time axis
so that the chosen time point is at zero), and a low-order polynomial is fitted
to data around it. Since polynomial approximation is more accurate for data
points closer in time, a weighting function is applied during the polynomial
estimation, which assigns weights to each data point based on its distance from
the point of interest. The value that the LPR predicts for the chosen time
point is then given by the intercept of the locally weighted polynomial at this
time point.

Formally this procedure can be expressed using the following set of equations:

\begin{equation} \label{eq:lpr_equations}
  \begin{aligned}
    y_t          & = f(t) + \epsilon_t                            \\
    \textbf{X}   & =
    \begin{bmatrix}
      1      & (t_1 - t^*)^1 & \dots  & (t_1 - t^*)^p \\
      \vdots & \vdots        & \ddots & \vdots        \\
      1      & (t_n - t^*)^1 & \dots  & (t_n - t^*)^p
    \end{bmatrix} \\
    \textbf{W}   & =
    \begin{bmatrix}
      w_{1, 1} &        &          \\
               & \ddots &          \\
               &        & w_{n, n}
    \end{bmatrix}                               \\
    \hat{f}(t^*) & =
    Intercept((\textbf{X}^T\textbf{WX})^{-1}\textbf{X}^T\textbf{Wy})
  \end{aligned}
\end{equation}

\noindent where a univariate process $f$ is inferred at the
chosen time point $t^*$. Then \textbf{X} is the model matrix of a multiple
linear regression, such that the columns correspond to polynomial
transformations up to degree $p$ of the data centered around
$t^*$\footnote{Note
  that the polynomial terms in this model matrix are derived from a Taylor
  series approximation around $t^*$ with $p$ derivative terms.}. Specifically,
\textbf{X} may contain polynomial transformations of centered time (i.e., the
first column is filled with ones, the second column contains the centered time
points, the third column the centered time points squared, and so on). Further,
\textbf{W} is a diagonal matrix containing the weights associated with each
datum. The last equation is a normal equation solving for the coefficients of a
weighted multiple linear regression. The intercept of this regression
gives the estimated value of the LPR at $t^*$.

The same procedure can be repeated to find the value that the LPR predicts for
the process at a different time point by centering the data around the new
point of interest. Since it is theoretically possible to repeat this process at
infinitely many time points, LPR is a non-parametric technique.
Figure~\ref{fig:locpol_dem} shows the estimated LPR for the damped oscillator
example process introduced in Figure~\ref{fig:examplar_npn}~(d). In this
figure, three truncated examples of local cubic regressions are shown in red,
which contribute to the overarching LPR for this process.\@

\begin{figure}[!t]
  \caption{Demonstration of a local polynomial regression}
  \fitfigure{locpol_demonstration.png}
  \figurenote{This figure shows how LPR (solid black) estimates the underlying
    process (dotted black). Here, three exemplary local cubic regressions (red)
    are shown that provide the values that the LPR predicts at the time points
    50, 100, and 150. The complete LPR was plotted by evaluating
    such local cubic regressions at 200 points along the time axis.}
  \label{fig:locpol_dem}
\end{figure}

When fitting an LPR, several decisions must be made regarding the optimal
weighting of the data and the degree of the local polynomial. The data
weighting in an LPR is achieved through a kernel function of the form

\begin{equation}
  w_{i, i} = K(\frac{t_i - t^*}{h})
\end{equation}

\noindent which, in this case, is a mathematical equation that determines the
influence of different data points during the local polynomial estimation.
Because these kernels are usually symmetric around the origin, they assign
weights to data points depending on how far away they are from the point of
interest $t^*$. Common choices for kernels include the Gaussian and
Epanechnikov kernels. Both assign higher weights to data points in closer
proximity. However, while the Gaussian kernel assigns small weights to all
distant data points, the Epanechnikov kernel assigns zero weights beyond a
certain distance. Additionally, the Epanechnikov kernel has also been shown to
be optimal in many applications, particularly when estimating points where the
entire width of the kernel lies within the data range
\parencite{fan_local_1997}. Each kernel is further defined by a bandwidth
parameter $h$, which determines the kernel's width and effectively controls the
influence of more distant data points. The bandwidth parameter represents the
wiggliness of the estimated process in practice. Several methods are available
to find the optimal bandwidth by optimizing a data-dependent criterion
function, such as the cross-validation error or the mean integrated squared
error \parencite{kohler_review_2014, debruyne_model_2008}. Most standard
software packages offer automated default procedures for this task. However,
the optimization criterion may also be selected based on specific research
objectives (e.g., optimizing a cross-validation error may be most attractive,
if the primary research interest is out of sample prediction).

In practice researchers also need to choose the degree of the local
polynomials, which additionally reflects an assumption about how smooth the
underlying process is. Specifically, for a first-degree LPR, the process should
not exhibit any corners, discontinuities, or vertical sections. This ensures
that the processes rate of change (i.e., derivative), approximated by the
first-order polynomial term, is well behaved. Higher-order local polynomials
require this smoothness for increasingly complex rates of change. For instance,
a second-degree LPR requires that the rate of change itself is smooth, in turn
ensuring that its rate of change is well behaved. This property should hold for
all $p$ rates of change of a process when using an LPR with $p$ degrees (under
Taylor's theorem). Typically, the degree of the local polynomials is chosen to
be low and odd. This choice reflects a bias-variance tradeoff, where
higher-order polynomials reduce bias but increase variance only when
transitioning from an odd to an even power
\parencite{ruppert_multivariate_1994}.

The second assumption made by the LPR is that the process has constant
wiggliness with respect to the chosen polynomial degree. This assumption is
made, because a single bandwidth parameter is used to capture the process at
any point in time. For example, a linear LPR, estimates the linear polynomials
using weights derived from a kernel with the same bandwidth at any point in
time. This bandwidth should then be optimal accross the entire process. This
assumption is less strict for higher order polynomials, due to their increased
flexibility and may be relaxed more by using a time-varying
bandwidth \parencite{fan_data-driven_1995} or polynomial degree
\parencite{fan_adaptive_1995}, but this extension is beyond the scope of this
paper.

Lastly, it is noteworthy, that unnless the underlying process follows a
polynomial trajectory of at most degree $p$ the approximation with local
polynomials is biased. For instance, whereas a process that follows a
quadratic trajectory can be accurately inferred at any point in time
by a local quadratic regression (and all higher order LPRs), a process that
follows an exponential trajectory cannot be inferred with perfect accuracy
by any LPR with a finite degree. Instead, there will be a small bias in the
estimate, that decreases for larger local polynomial degrees.
However, this bias is usually negligible and there are methods available to
correct for it \parencite{R-nprobust}.

\subsection{Gaussian process regression}

The second non-parametric technique is Gaussian process (GP) regression, a
Bayesian approach that directly defines a probability distribution over an
entire family of non-linear functions, which is flexible enough to capture many
complex processes effectively \parencite{rasmussen_gaussian_2006,
  betancourt_robust_2020, roberts_gaussian_2013}. Unlike regular probability
distributions (e.g., normal distribution) that specify the likelihood of
single values, Gaussian processes determine how likely entire
(non-linear) functions are. A GP is defined indirectly, such that, if the
functions it describes are evaluated at any finite set of time points, the
resulting sample of function values would follow a multivariate normal
distribution. In a Bayesian framework, one can use a GP to define a prior
distribution for the latent process, as $P(f) \sim GP$. This prior is then
combined with an appropriate likelihood for the observed data to obtain a
posterior distribution for the latent process given the observed data.

\begin{equation}
  P(f \, | \, \textbf{y})  \propto P(\textbf{y} \, | \, f) P(f)
\end{equation}

\noindent This posterior distribution represents an updated belief about which
functions describe the latent process well \parencite{kruschke_doing_2011},
allowing one to draw inferences about the process.
Figure~\ref{fig:gp_dem} illustrates such a posterior distribution for the
running example process. The red lines represent a sample of non-linear
functions drawn from the posterior distribution, such that the pointwise
average of these functions provides a mean estimate for the underlying process.

\begin{figure}[!t]
  \caption{Demonstration of a Gaussian process regression}
  \fitfigure{gp_demonstration.png}
  \figurenote{This figure shows how a Gaussian process regression estimates the
    underlying process (dotted black). Here, a sample of functions drawn from
    the posterior Gaussian process probability distribution with a squared
    exponential kernel is shown (red). The predicted value for the underlying
    process is then obtained by averaging the drawn functions.}
  \label{fig:gp_dem}
\end{figure}

The GP prior is parameterized by a mean function $m(t)$ and a covariance
function $cov(t, \, t)$, which are continuous extensions of the mean vector and
covariance matrix of a multivariate normal distribution. These functions can be
selected based on domain knowledge or through data-driven model selection
\parencite{richardson_gaussian_2017, abdessalem_automatic_2017}. In practice,
the mean function is often set to zero when no specific prior knowledge is
available. This does not constrain the posterior mean to zero but instead
indicates a lack of prior information about its deviations from zero. The
covariance function is typically based on a kernel function, which assigns
covariances between time points depending on their distance (e.g.,
quadratic exponential, Matern class kernels), such that

\begin{equation}
  cov(t_i, t_j) = k(|t_i - t_j|)
\end{equation}

Lastly, the mean and covariance functions also include parameters, called
hyperparameters, which can be estimated by assigning appropriate priors to
them. Most common covariance kernels build on a characteristic lengthscale and
a marginal standard deviation parameter. The characteristic lengthscale
effectively determines the wiggliness of the estimated process, by quantifying
how quickly the covariance decreases with increasing distances between time
points. The marginal standard deviation describes the spread of the functions
which are described by the GP at any point in time. However, GP regression can
also include additional hyperparameters, allowing for more specific theories to
be tested through model comparison. Most standard software provides general
default GP priors. However, it is also possible to choose or construct a
specific GP prior to fit a specific context or research objective.

Whereas LPR is a mainly data driven procedure GP regression is more model
based. The GP prior generates a family of functions to which the process is
assumed to belong. This makes it possible to include theoretical domain
knowledge and specific hypotheses of interest in the kinds of function that are
modelled by a GP\@. If one for example expects that a non-linear process has
a general upwards trend, it would be possible to model a linear mean function
with non-linear deviations that are captured by a GP\@. However,
this also means that to accurately capture a process, it is important that the
functional family generated by a GP is similar to the actual process. Most
common choices for covariance kernels for example result in smooth and
covariance-stationary GPs with constant wiggliness. Another difference between
GP regression and LPR is that the Bayesian estimation underlying the GP
regression provides a natural approach to uncertainty quantification. While
there is no direct way of quantifying the uncertainty associated with the
bandwidth of the LPR, the uncertainty of the lengthscale of the GP is captured
in its posterior distribution.

\subsection{Generalized additive models}

Generalized additive models (GAM) are a class of semi-parametric models that
build on so called smooth terms, which are non-linear functions that are
inferred from the data through smoothing splines
\parencite{wood_generalized_2006, wood_inference_2020,
  hastie_generalized_1999}. Smoothing splines extend regular spline regression
to
address the knot placement problem. Instead of placing knots at meaningful
locations, smoothing splines use a very large number of knots to guarantee
overfitting the data. One approach to this is to use cubic splines, which are
piecewise cubic polynomials. By placing a knot at each data point these splines
perfectly interpolate the data. Alternatively, thin-plate splines can be used,
which entirely avoid knots and instead utilize increasingly wiggly basis
functions that are defined over the entire range of the data
\parencite{wood_thin_2003}. These basis functions are then combined in a
regression model similar to polynomial regression (i.e., each basis function
functions as predictors in a regression equation). Using as many basis
functions as data points is analogous to placing a knot at each data point. To
prevent the resulting overfitting, smoothing splines add an additional penalty
term, similar to those used in a lasso or ridge regression, to control how
closely the smooth term fits the data during the estimation
\parencite{gu_smoothing_2013, wahba_spline_1980}. This penalty balances the
complexity and fit of the smooth term, ensuring the model captures the
underlying process accurately. The optimal weight of the penalty is typically
determined by minimizing a criterion function, such as the generalized
cross-validation criterion or thorough likelihood maximization, which is
performed automatically in standard software \parencite{wood_generalized_2006,
  golub_generalized_1997}.

A smoothing spline for a single smooth term $\beta$ may then be written as

\begin{equation}
  \begin{aligned}
    \hat{\beta}(t) & = \argmin_\alpha \, P(\textbf{y} \, | \, \beta(t)) +
    \lambda \int {(\beta(t)'')}^2 dt                                      \\
    \beta(t)       & = \sum^K_{k = 1} \alpha_k R_k(t)
  \end{aligned}
\end{equation}

\noindent where the first part of the equation describes the likelihood
of the data given the smooth term and the second part of the equation
corresponds to the penalty term. This illustrates, how the smoothing spline
balances data fit, in the form of the likelihood, and the complexity or
wiggliness of the estimate, measured by the overall squared curvature of the
curve. Here, $\lambda$ denotes the weight assigned to the penalty term that is
optimized over. Lastly, the smoothing spline is
comprised of spline basis functions $R_k(t)$, such as the introduced cubic or
thin-plate splines and their respective regression coefficients $\alpha_k$.

GAMs extend on the smoothing spline approach by making it possible to combine
multiple smooth terms (of potentially different input variables) in an
overarching additive regression model, where each
smooth term essentially functions as a predictor within a regular regression
analysis. This approach makes it possible to formulate models such
as a time-varying autoregressive model, where the intercept and autoregressive
parameters are smooth terms of time \parencite{bringmann_changing_2017,
  bringmann_modeling_2015}. By integrating non-parametric smooth terms into a
broader parametric model, GAMs become semi-parametric models that are
well-suited for testing specific hypotheses while keeping the flexibility
needed to accurately capture the latent process. Figure~\ref{fig:gam_dem}
illustrates a simple GAM construction with a single smooth term for time,
fitted to the example process. The first ten thin-plate smoothing spline basis
functions of the nearly 200 basis functions that make up the smooth term are
shown in red.

\begin{figure}[!t]
  \caption{Demonstration of the construction of a GAM}
  \fitfigure{gam_demonstration.png}
  \figurenote{This figure shows how generalized additive models (solid black)
    estimate the underlying process (dotted black). Here, the predicted values
    for the process at any point in time correspond to the weighted average of
    the basis functions (red).}
  \label{fig:gam_dem}
\end{figure}

Similar to GPs, GAMs constitute a more model based approach to inferring
non-linearity. Indeed, GAMs and GP regression are closely related techniques,
as each GAM can be understood as a (unconventional) GP with a partially
improper prior\footnote{Any basis function regression with Gaussian priors for
  the coefficients is a GP in weight space. The smoothing spline underlying the
  GAMs is such a basis function regression in which the coefficient priors
  serve
  the same purpose as the penalty term. Due to the null space of the penalty
  term, some of these priors are generally going to be imporper.}. These
improper
priors make the GAMs more flexible to model processes, which might be outside
the functional family generated by a standard GP construction
\parencite{wahba_improper_1978}. In addition to this and similar to GPs, GAMs
offer an intuitive way to combine partial theories with data driven smooth
terms. In this way, GAMs can also, for instance, model a linear function with
non-linear deviations, which are captured by a smooth term. GAMs also provide
an estimate of the wiggliness of the process through the effective degrees of
freedom of the model. These are a measure of the models effective complexity,
where an effective degree of freedom of one corresponds to a linear model.
However, in contrast to LPR and the standard GP construction, GAMs do not
assume constant wiggliness. Unfortunately, GAMs do not provide a measure of the
uncertainty of the effective degrees of freedom (EDF). Lastly, it is possible
to
interpret the basis function coefficients in a GAM\@. The interpretation of
these coefficients does however depend on the chosen basis functions and on the
other terms in the model. Table~\ref{tab:meth_sum} summarizes the
similarities and differences between the three introduced methods.

\begin{table}[htbp]
  \begin{center}
    \begin{threeparttable}
      \caption{A comparison of LPR, GP regression and GAMs}
      \label{tab:meth_sum}

      \begin{tabularx}{\linewidth}
        {>{\raggedright}s
          >{\raggedright}X
          >{\raggedright}X
          >{\raggedright\arraybackslash}X}
        \toprule
                                                                        &
        \multicolumn{1}{c}{LPR}                                         &
        \multicolumn{1}{c}{GP}                                          &
        \multicolumn{1}{c}{GAM}
        \\
        \midrule
        Advantages                                                      &
        \begin{itemize}
          \item Intuitive theory
          \item Completely data driven
        \end{itemize}                                    &
        \begin{itemize}
          \item Most interpretable parameters
          \item Natural uncertainty quantification
          \item Flexible modelling framework can incorporate prior theory
        \end{itemize} &
        \begin{itemize}
          \item Intuitive theory
          \item Some interpretable parameters
          \item Flexible modelling framework can incorporate prior theory
        \end{itemize}
        \\ \midrule
        Disadvantages                                                   &
        \begin{itemize}
          \item Least interpretable parameters
          \item Biased for most processes
          \item No uncertainty estimate for the wiggliness
        \end{itemize}                &
        \begin{itemize}
          \item Unintuitive theory
          \item Difficult to specify in practice
        \end{itemize}                          &
        \begin{itemize}
          \item No uncertainty estimate for the wiggliness
        \end{itemize}
        \\ \midrule
        Key assumptions                                                 &
        \begin{itemize}
          \item P-times differentiable process
          \item Constant wiggliness
        \end{itemize}                            &
        \begin{itemize}
          \item Assumptions depend on chosen specifications
        \end{itemize}               &
        \begin{itemize}
          \item Smooth process
          \item Homoscedasticity
        \end{itemize}
        \\ \midrule
        Estimation                                                      &
        \begin{itemize}
          \item OLS
        \end{itemize}                                                 &
        \begin{itemize}
          \item Bayesian
        \end{itemize}                                                 &
        \begin{itemize}
          \item OLS, MLE, Bayesian
        \end{itemize}
        \\ \midrule
        Key sources of information                                      &
        \begin{itemize}
          \item \textcite{fan_local_2018}
        \end{itemize}                                 &
        \begin{itemize}
          \item \textcite{rasmussen_gaussian_2006}
        \end{itemize}                        &
        \begin{itemize}
          \item \textcite{wood_generalized_2006}
        \end{itemize}
        \\
        \bottomrule
      \end{tabularx}
    \end{threeparttable}
  \end{center}
\end{table}

\section{Simulation} \label{simulation}

\subsection{Problem}

A simulation study was conducted to assess the effectiveness of the introduced
methods in recovering different non-linear processes, which may be encountered
in EMA research (Figure~\ref{fig:examplar_npn}). In this simulation, the three
methods were not only compared against each other, but also to a polynomial
regression model (which is the current most used method to model non-linear
trends in psychology) and to parametric models that accurately specify the
non-linear process. These (data generating) parametric models were added to
serve as a benchmark for non-linear process recovery. To apply the introduced
methods under the conditions described in the introduction, and within the
constraints of available software implementations, the simulation focused on a
univariate single-subject design. Hence, the simulated data represented
repeated measurements of a single variable for one individual.

\subsection{Design and Hypotheses}

To conduct the simulation with processes that might be encountered in real EMA
studies, we selected the exemplar processes illustrated in
Figure~\ref{fig:examplar_npn} as a basis. These include two growth curves,
modeled as an exponential and a logistic growth curve, a mean-level switching
process, modeled as a cusp catastrophe, and a self-regulatory process,
represented by a damped oscillator. For each of these processes,
we varied the sampling period (i.e., how long each process is measured),
the sampling frequency (i.e., how often each process is measured), and the
smoothness of the process. Afterwards, each of the introduced non-parametric
methods, as well as a polynomial regression and parametric models were used
infer the underlying processes from the data.

For the LPR and the GP, we expect that both method, using the configurations in
which they are most often applied and implemented in standard software, will
struggle to infer processes that are not continuous (i.e., with sudden jumps),
have varying wiggliness, and are not smooth (i.e., differentiable). We expect
this, because both methods by default produce continuous, smooth estimates with
a single constant bandwidth or lengthscale. For the GAMs, we expect that only
the continuity and smoothness of the process  will influence the performance,
since GAMs do not assume constant wiggliness. The parametric modeling approach
is expected to provide the most accurate inferences, serving as a benchmark for
comparison with the other methods.

First, we hypothesize that the cusp catastrophe model, which is the only
process featuring jumps, will be least accurately inferred by all methods.
Second, all four processes exhibit changes in wiggliness (as defined by each of
the non-parametric methods respectively) over time. However, while the
wiggliness of the exponential and logistic growth functions and the damped
oscillator decreases monotonically, the cusp catastrophe's wiggliness changes
cyclically (i.e., low wiggliness during the plateau phases and very high
wiggliness during the jumps). Therefore, we hypothesize that longer sampling
periods for the exponential and logistic growth curves and the damped
oscillator will reduce the inference accuracy of the LPR and the GP, as the
single bandwidth or lengthscale parameter becomes increasingly inadequate to
capture the changing wiggliness over time. We do not expect this effect to
occur for the cusp catastrophe process, or when using GAMs.

To manipulate the smoothness of the processes a dynamic error component was
added to each process. These dynamic errors reflect external perturbations to
the latent construct that are carried forward in time. For instance, if a
participant experiences an unusually pleasant conversation that elevates their
true positive affect, this change represents an error effect if it is not
accounted for by the model. However, since the true positive affect level has
increased, this will influence future measurements due to for example emotional
inertia. To add these errors, each process was perturbed by a normally
distributed error at each point in time, resulting in non-smooth (i.e.,
non-differentiable or rough) trajectories. The degree of roughness was
controlled by the variance of these dynamic errors and we considered variances
of 0.5, 1, and 2 reasonable relative to the process range.
Figure~\ref{fig:exemplar_pn} illustrates one possible realization of the
exemplar processes with dynamic errors. Importantly, we intentionally omitted a
condition without dynamic noise from this simulation, as dynamic noise is
reasonably expected to be present in all psychological intensive longitudinal
data (ILD).

\begin{figure}[!t]
  \caption{Non-linear exemplar processes with dynamic errors}
  \fitfigure{exemplar_process_noise.png}
  \figurenote{This figure shows one possible realization of how the exemplar
    processes could unfold with dynamic errors. The processes have dynamic
    error standard deviations of 0.25 (left) and 0.5 (right).}
  \label{fig:exemplar_pn}
\end{figure}

Additionally, the sample size was varied during the simulation by manipulating
both the sampling period and the sampling frequency, as these distinct
methodological choices are expected to impact the performance of the analysis
methods differently. Specifically, for the LPR and the GP, which rely only on
data in local neighborhoods during the estimation, we expected that extending
the sampling period beyond this neighborhood will not increase the inference
accuracy. In fact, if the process exhibits changing wiggliness over the
extended period, as previously discussed, increasing the sampling period might
even negatively affect the inference accuracy. In contrast to this, we expected
the GAMs, which incorporate the entire dataset in their estimation, to perform
better with a longer sampling period. Since within the simulation there is no
inherent scaling to the time variable, we simulated either only the first half
of each process or the entire process to represent different sampling periods.
For the ease of reading and to correspond to typical EMA conditions, this will
be referred to as sampling over either one or two weeks. However, this scaling
is arbitrary and could be changed to any other time frame. Lastly, we expected
that increasing the sampling frequency will generally improve the inference
accuracy across all methods, since there is more information about the latent
process available. Relative to the introduced weekly scale, we tested sampling
frequencies of three, six, and nine measurements per day, to cover typical EMA
sample sizes \parencite{wrzus_ecological_2023}.

\subsection{Outcome measures}

To evaluate and compare the performance of the different analysis methods, we
focused on three outcome measures. The first two assessed each method's
accuracy in predicting the process values at or between the observed time
points. These predictive accuracy measures indicate how well each method
captures the underlying non-linear process. The third outcome measure evaluated
the accuracy of the uncertainty estimates provided by each method.
Specifically, whether the confidence or credible intervals produced by each
method correctly included the true state value the expected proportion of
times.

\subsubsection{Capturing the non-linear process}

To assess how effectively each method captured the non-linear process at the
observed time points, we calculated the mean squared error (MSE) between the
estimated and generated process values. Additionally, to evaluate how well each
method would predict omitted process values within the process range, we
computed the generalized cross-validation
\parencite[GCV;][]{golub_generalized_1979} criterion for each method and data
set. The GCV is a more computationally efficient and rotation-invariant version
of the ordinary leave-one-out cross-validation criterion, with the same
interpretation. Both, the GCV and the leave-one-out cross-validation criterion,
describe how accurately the model predicts omitted data points within the
design range, which provides information about how well the model interpolates
the process.

The results of the mean MSE and GCV values in each condition will be presented
in figures. Additionally, to efficiently summarize the high dimensional results
and formalize the analysis, two separate ANOVAs were fit to the MSE and GCV
values. Both ANOVAs included all possible main and interaction effects of the
data generating processes, the analysis methods, and simulation conditions
(i.e., sampling period, frequency, and dynamic error variance). Although, the
parametric models were excluded from the ANOVAs as they were expected to
outperform and hinder the comparison of the other methods. Since even very
small effects can lead to statistical significance at the large sample
sizes considered in this simulation, we focussed on the effects that showed at
least a small effect size according to the partial $\eta^2$ (> 0.01). The
assumptions of the ANOVAs were tested. However, given the large sample sizes in
this simulation, the ANOVAs were assumed to be robust to moderate violations of
normality \parencite{blanca_non-normal_2017}, and any potential violations of
homoscedasticity were addressed using heteroscedasticity-consistent standard
errors.

\subsubsection{Uncertainty quantification}

To evaluate the uncertainty estimates provided by each method, we recorded
whether the true generated process was located within the confidence or
credible intervals at each time point. Subsequently, the average confidence
interval coverage proportion for each method and data set was obtained, by
averaging over all time points. Given that all confidence or credible intervals
were set at a 95\% confidence level, the expected coverage proportion should
ideally be close to 95\%. Due to Monte Carlo error in the simulation
($\max(se_{CIC}) \approx 0.03$) individual average confidence interval
coverages are expected to deviate from the ideal 95\%. Because of this, average
coverage proportions between 89\% and 100\% were also deemed acceptable.
Average coverage proportions below 89\% indicated either a poor approximation
of the underlying process or underestimated uncertainty.

\subsection{Data generation}

Each process was represented as a generative stochastic differential
equation model. These dynamic models describe the relationship between the
process's current value and its instantaneous rate of change. This can be
combined with information about the initial state of the process to infer the
entire trajectory indirectly. For instance, the stochastic differential
equation model used to represent the introduced logistic growth process can be
expressed as follows:

\begin{equation} \label{eq:2}
  dy = r y (1-\frac{y}{k})dt + \sigma dW_t
\end{equation}

\noindent The first half of this model defines the deterministic dynamics of
the process. It relates the rate of change of $y$ to its current value and to
how far away the current value is from the asymptote $k$ through a growth rate
constant $r$. The second part of the model accounts for the dynamic errors in
the form of a Wiener process. The Wiener process is a continuous
non-differentiable stochastic process, which describes normally distributed
dynamic errors over any given discrete time interval. These errors have a mean
of zero and a variance depending on the length of the time interval and
$\sigma$, making them optimal for this simulation. Importantly, these dynamic
errors continuously influence the rate of change of the process and are
propagated forward in time through the deterministic dynamics of the model.

All four processes were modeled as stochastic differential equations by
substituting their respective deterministic dynamics into the first half of
Equation~\ref{eq:2}, as detailed in the online supplementary material. Latent
process data were simulated using the Euler-Maruyama method, which approximates
stochastic differential equation systems with an arbitrarily high accuracy by
linearizing them over small discrete time intervals. The resulting
high-resolution data were then subsampled to achieve the desired sampling
frequency. Finally, measurement errors were added to the latent process data at
each time point from a standard normal distribution, generating the final sets
of observations,

Based on an initial pilot sample of 30 data sets per condition, we determined
the number of replications needed to achieve a monte carlo standard error of
less than 0.03 for the confidence interval coverage
\parencite{siepe_simulation_2023}. These Monte Carlo standard errors reflect
the expected variation in the outcome statistics due to random processes within
the simulation. This analysis showed that 100 replications per condition
would result in maximal Monte Carlo standard errors of approxmiately $se_{MSE}
  \approx 0.05$, $se_{GCV} \approx 0.38$, and $se_{CIC} \approx 0.03$.

\subsection{Model estimation}

After simulating the data, all introduced methods were applied to each data set
using the statistical software R \parencite{R-base}. First, the LPR was
estimated using the nprobust package \parencite{R-nprobust}, which allows to
correction for the bias inherent in LPRs. Second, GPs were estimated in STAN
\parencite{R-cmdstanr} with a zero mean and a squared exponential covariance
function, following common practice. Third, GAMs with a single smooth term for
time were fitted using the mgcv package \parencite{R-mgcv_a}. The polynomial
regressions were estimated using base R, with correlated (i.e.,
standard) polynomial terms. Finally, the
parametric differential equation models corresponding to the true
data-generating models were estimated using the Dynr package
\parencite{R-dynr}. After fitting each model to the data, they were used
to obtain point and interval estimates (i.e., 95\% confidence and credible
intervals) for the latent process at each time point. A detailed description of
each model fitting procedure is provided in the online supplementary material.

To ensure reliable model fit and reasonable inferences, the fitting procedures
for each method were validated on pilot samples within each condition. After
this, an initial run of the simulation was performed, which revealed that the
GAMs over- or linearly underfit some data sets and that the parametric models
overfit some data sets. To prevent this, the fitting procedures of both methods
were adjusted and the simulation rerun. Further, if any models failed to
converge during the simulation, the corresponding outcome measures were
excluded from the following analyses.

\subsection{Results}

In the simulation some of the GP regressions and parametric models did not
converge. This is most likely due to the small sample sizes considered and the
automated model fitting in the simulation. Most notably, the parametric models
were not able to infer the cusp catastrophe from the small simulated data sets
due to the complexity of the model. The performance measures of the methods
that did not converge were removed from the following analysis. Additionally,
the parametric models did overfit for a small number of data sets. The
complete simulation results and data are available in the online supplementary
material.

\subsubsection{Capturing the non-linear process}

First, it is noteworthy that all considered methods visually performed well in
mean predicting the simulated processes. Figure~\ref{fig:smooth} illustrates an
example from each process being inferred by all methods. The predicted means
produced by each method closely followed the simulated processes, although the
LPR appears to underfit for some data sets. Additionally, near the boundaries
(i.e., the ends) of the simulated time series, the GP regression sometimes
tends towards zero, which is a characteristic of the squared exponential kernel
(this can be seen by comparing the GP inference in Figure~\ref{fig:gp_dem} to
the estimates in Figure~\ref{fig:gam_dem} or~\ref{fig:locpol_dem}).
\textit{Further,
  the polynomial regression appears to overfit near the boundary, resulting in
  excessive uncertainty, which is a known behavior of polynomial regressions.}
However, there is considerable variation and overlap in the accuracy of the
different methods across the various data sets, which highlights the need for a
more formal and objective analysis of the performance of each method.

\begin{sidewaysfigure*}[htbp]
  \caption{Example processes inferred by each of the introduced methods}
  \fitfigure{smooth.png}
  \label{fig:smooth}
  \figurenote{This figure shows how each of the introduced methods inferred an
    example of each of the processes from the simulation.}
\end{sidewaysfigure*}

To achieve this, two separate ANOVAs were fitted to the MSE and GCV values,
including all possible main and interaction effects. Although the residuals for
both models showed considerable deviations from normality, which were mainly
characterized by being leptokurtic, the residuals were unimodal and
approximately symmetric. Given the large sample sizes in this simulation, we
thus assumed that the ANOVAs were robust to these deviations. Further, a
Breusch-Pagan test indicated heteroscedasticity in the residuals for both
outcome measures, which was corrected for using heteroscedasticity-consistent
standard errors. Lastly, Bonferroni corrections were applied to adjust the
p-values for conducting two separate ANOVAs.

The type-III ANOVAs for both outcome measures indicated that all main and first
order interaction effects were significant. Additionally, some higher order
interaction terms were significant, which differed between the two outcome
measures. However, due to the large sample size, even very small effects can
lead to statistical significance. Therefore, Table~\ref{tab:peta} presents all
effects for which the partial-$\eta^2$, which gives the proportion of variance
explained by an effect after partialling all other effects out, indicates at
least a small effect size for either the MSE or the GCV\@. The following
sections will focus on describing some of these effects and a comprehensive
overview of all effects in the models can be found in the online supplementary
material.

\begin{table}[tbp] % import(tables/eta_squared.txt)

  \begin{center}
    \begin{threeparttable}
      \caption{Effect sizes from the MSE and GCV ANOVAs}
      \label{tab:peta}
      \begin{tabular}{lll}
        \toprule
        Effect                & \multicolumn{1}{c}{partial-$\eta^2$ MSE} &
        \multicolumn{1}{c}{partial-$\eta^2$ GCV}
        \\
        \midrule
        Method                & 0.40                                     & 0.26
        \\
        Process               & 0.55                                     & 0.41
        \\
        SP                    & 0.18                                     & 0.09
        \\
        SF                    & 0.15                                     & 0.24
        \\
        DEV                   & 0.56                                     & 0.49
        \\
        Method:Process        & 0.23                                     & 0.13
        \\
        Method:SP             & 0.18                                     & 0.11
        \\
        Process:SP            & 0.07                                     & 0.05
        \\
        Method:SF             & 0.05                                     & 0.01
        \\
        Process:SF            & 0.01                                     & 0.05
        \\
        Method:DEV            & 0.16                                     & 0.09
        \\
        Process:DEV           & 0.27                                     & 0.24
        \\
        SP:DEV                & 0.02                                     & 0.03
        \\
        SF:DEV                & 0.01                                     & 0.05
        \\
        Method:Process:SP     & 0.08                                     & 0.07
        \\
        Method:Process:SF     & 0.02                                     & 0.01
        \\
        Method:Process:DEV    & 0.07                                     & 0.05
        \\
        Method:SP:DEV         & 0.02                                     & 0.03
        \\
        Process:SP:DEV        & 0.01                                     & 0.01
        \\
        Process:SF:DEV        & 0.01                                     & 0.02
        \\
        Method:Process:SP:DEV & 0.02                                     & 0.02
        \\
        \bottomrule
      \end{tabular}
      \tablenote{This table shows all effects from the MSE and GCV ANOVA that
        had at least a small effect partial-$\eta^2 >= 0.01$ on either
        outcome. SP\@: Sampling period; SF\@: Sampling frequency; DEV\@:
        Dynamic error variance.}
    \end{threeparttable}
  \end{center}

\end{table}

Figure~\ref{fig:mean_results_mse}~(a) illustrates the mean MSE values with
which each method inferred each process, averaged across sampling periods,
frequencies, and dynamic error variances. This figure highlights the main
effect of the analysis method, as there are clear differences in the mean MSE
with which each method inferred all processes. Specifically, the parametric
modelling showed the lowest average MSE, followed closely by the GAMs, whereas
the GP regression, LPR, and the polynomial regression had larger average MSEs.
Additionally, Figure~\ref{fig:mean_results_mse}~(a) illustrates the main effect
of the processes, as each process was inferred by all the methods with
different mean MSE values. Most notably, the cusp catastrophe was inferred with
lower MSE values than the other processes by all methods. Further, one can see
that there is an interaction between the analysis methods and the processes,
since the differences in how accurately each process was inferred differ
between the methods. For example, the difference between the MSEs for the cusp
catastrophe and the MSEs for the other processes is larger for the LPR than for
the other considered methods. Notably, the polynomial regression displayed a
larger mean MSE for all processes except the cusp catastrophe in comparison to
the more advanced statistical methods.

\begin{sidewaysfigure*}[htbp]
  \caption{Mean MSE effects across all processes, analysis
    methods, and simulation conditions}
  \fitfigure{mean_results_mse.png}
  \figurenote{Panel (a) shows the effect of the analysis method for each
    process. The other three panels show the effects of measurement
    period (b), measurement frequency (c), and dynamic error variance (d) for
    each analysis method and latent process.}
  \label{fig:mean_results_mse}
\end{sidewaysfigure*}

Figure~\ref{fig:mean_results_mse}~(b) shows the average MSE over different
measurement periods for each analysis method and process, averaged over
measurement frequencies and dynamic error variances. The results indicate that
sampling over the entire process, rather than just the first half, led to
higher average MSE values for both local and global polynomial regression
across all processes. This effect was much less pronounced for the GP
regression, absent for the GAMs, and reversed for the parametric models.
Further, Figure~\ref{fig:mean_results_mse}~(c) illustrates that the mean MSE
generally decreased with larger sampling frequencies for each method and
process, while averaging over the sampling periods and dynamic error variances.
Lastly, Figure~\ref{fig:mean_results_mse}~(d) demonstrates that larger dynamic
error variances increased the mean MSE values across all methods and processes,
when averaged over sampling periods and frequencies. However, this effect was
least pronounced for the cusp catastrophe.

Figure~\ref{fig:mean_results_gcv} displays the corresponding effects for the
mean GCV values. Similar to the MSE results, the GAMs show a mean GCV value
closest to the benchmark parametric models. However, the GPs also display
comparable performance to the GAMs in terms of the GCV\@. The local and
global polynomial regressions show considerably larger mean GCV values for
all processes except the cusp catastrophe. The effects of
measurement period, frequency, and dynamic error variance on the mean GCV
appear to follow largely the same patterns that were observed for the mean
MSE\@.

\begin{sidewaysfigure*}[htbp]
  \caption{Mean GCV effects across all processes, analysis
    methods, and simulation conditions}
  \fitfigure{mean_results_gcv.png}
  \figurenote{Panel (a) shows the effect of the analysis methods for each
    latent process. The other three panels show the effects of measurement
    period (b), measurement frequency (c), and dynamic error variance (d) for
    each analysis method and latent process.}
  \label{fig:mean_results_gcv}
\end{sidewaysfigure*}

\subsubsection{Uncertainty quantification}

Figure~\ref{fig:mean_results_ci_coverage} shows the average confidence interval
coverage proportion for the conditions described above. The grey area
represents an average confidence interval coverage between 89\% and 100\%,
which indicates no considerable deviation from the ideal 95\% given the Monte
Carlo error of the simulation. Only the parametric models produced mean
confidence interval coverages that fell within this area. Among the other
methods, the GAMs, produced the largest average confidence interval coverage,
followed by the global and then local polynomial regression. The GP regression
appears to result in the smallest confidence interval coverage.

\begin{sidewaysfigure*}[htbp]
  \caption{Average confidence interval coverage across all processes, analysis
    methods, and simulation conditions}
  \fitfigure{mean_results_ci_coverage.png}
  \figurenote{Panel (a) shows the effect of the analysis methods for each
    latent process. The other three panels show the effects of measurement
    period (b), measurement frequency (c), and dynamic error variance (d) for
    each analysis method and latent process.}
  \label{fig:mean_results_ci_coverage}
\end{sidewaysfigure*}

\subsection{Conclusion}

This simulation showed that among the considered methods the GAMs inferred all
processes with the most accuracy as indicated by the mean MS, GCV and
confidence interval coverage. The GAMs performed closest to the true data
generating models, followed by the GP (with regards to the MSE and GCV, since
the GP had the lost confidence interval coverage), and then the local and
global polynomial regression. This result is unexpected, as we anticipated that
the smooth inferences produced by GAMs may be ill-suited for inferring the
rough (i.e., non-differentiable) processes in the simulation. However, as
Figure~\ref{fig:smooth} shows, all methods except the parametric models
produced smooth inferences. This suggests that the additional flexibility that
the GAMs provide over the LPR, by not assuming constant wiggliness, and over
the GP, by effectively relying on improper priors, enabled the GAMs to infer
the processes more accurately in the simulation. From this follows, that GAMs
are an attractive starting point for modelling unknown processes in practice.
Especially, when there is little prior theory about the functional form of the
process.

However, it is important to note that the observed results may be most
attributable to the specific configurations that were used rather than the
general modelling approaches underlying each method. The specific
configurations of each method were chosen to reflect how each method is most
commonly applied in practice and not to optimally infer the simulated
processes. Consequently, different configurations and extensions are likely to
improve the performance of the LPR and GP\@. Especially, since the GAMs used in
this simulation correspond to a GP with a linear mean function (with improper
priors) and a non-stationary covariance function
\parencite{wahba_improper_1978, rasmussen_gaussian_2006}. This implies that
there exists a GP configuration which performs at least as well as the GAMs.
While such a configuration would heavily deviate from how GPs are most commonly
applied, common changes like using a Matern class kernel, which makes less
strict smoothness assumption about the data, could also be expected to improve
the accuracy of the GPs. Therefore, if there is already some prior knowledge
about the form of the process available, GPs are still a very interesting
modelling approach in practice, as they yield more interpretable inferences.
However, this likely still requires fine tuning the configurations of the
GP to one's specific conditions.

Regarding the LPR, several optimality results have been found indicating that
LPR should be at least as accurate as GAMs and GPs for p-times differentiable
processes \parencite{fan_local_1997}. It is however unclear whether these
results generalize to the non-differentiable processes that we suspect can be
found most often in psychology. Lastly, the tested polynomial regression
inferred the underlying processes least accurately. This appeared to be due to
numerical model instability that resulted in underfitting, since not enough
higher-order terms could be included in the models, and due to divergence
towards the ends of the timeseries. The numerical model instability can be
mitigated by using orthogonal polynomials, which are however more difficult to
interpret. Generally, there appears to be no reason to use a global polynomial
regression instead of a GAM, in any situation in which prior theory does not
strongly suggest that the process follows a low order polynomial trajectory.

Contrary to our expectation, the simulation indicated that the cusp
catastrophe process was inferred most accurately by all methods. We
had anticipated that the smooth, continuous estimates produced by these methods
would struggle to adapt to the apparent jumps exhibited by this process.
However, this effect seems to have been overshadowed by the cusp catastrophes
strong resilience to external perturbations. This property is highlighted in
Figure~\ref{fig:exemplar_pn}, where dynamic errors with the same variance have
been applied to all four processes. Despite the perturbations being of equal
variance, the cusp-catastrophe model appears to be the least affected and even
closely resembles the unperturbed process (Figure~\ref{fig:examplar_npn}).
Further evidence of this can be seen in the simulation, where the effect of
increasing the dynamic error variance was weakest for the cusp process. Due to
this, the simulation was rerun with considerably smaller dynamic error
variances, and under these conditions, the cusp model was indeed inferred with
the least accuracy.

% Still need to add citations to this paragraph
The results further indicate that measuring at a higher frequency increased the
inference accuracy of all considered methods. Therefore, it is generally
advantageous from a statistical point of view to measure as frequently as
possible. However, in practice, this must be balanced against considerations
such as participant burden and fatigue, which can adversely affect data quality
if measurements are taken too often. Similarly, when selecting the sampling
period, it is essential to use domain knowledge about the scale of the
underlying dynamics to ensure that the measurements capture sufficient
variation in the latent process. Beyond this, the simulation showed that
extending the sampling period improved the inference accuracy of the parametric
models but may decrease the accuracy of the LPR and the polynomial
regression\@. This reduction in accuracy could however possibly be mitigated by
using extensions for a variable bandwidth, or polynomial degree in an LPR\@.
Lastly, the simulation revealed that larger dynamic error variances decreased
the accuracy of all methods. Therefore, reducing the magnitude of dynamic
errors is advisable in practice. This could, for example, be achieved by
measuring context variables and other sources of perturbations and
incorporating them into the model.

\section{An Empirical Example} \label{empirical_example}

In the following, we applied GAMs to depression data from the Leuven clinical
study, which were obtained from the EMOTE database
\parencite{kalokerinos_emote_nodate}. We selected the data for their
heterogeneous sample, which contains momentary depression scores of
participants who met the DSM criteria for mood disorders or borderline
personality disorder during an intake, as well as depression scores for healthy
controls. For a more thorough sample and data description, including screening
protocols, see \textcite{heininga_dynamical_2019}. The diversity in the study
population makes it likely to find non-linear processes for at least some
participants. For example, emotional inertia, the extent to which an emotional
state carries over across time points, has been shown to vary within
individuals over time \parencite{koval_changing_2012}, which makes it likely
that the processes underlying these data are non-stationary and thus
non-linear.

The dataset used for this application contained 77 participants in the clinical
sample and 40 participants in the control sample, who were matched to the
clinical sample by gender and age, resulting in a total sample size of
117\footnote{The original published data set contained one additional
  participant who was removed for this analysis since they had a depression
  score
  of zero across all assessments.} 3. The participants completed seven days of
semi-random EMA assessments, with 10 equidistant assessments per day. Note that
the starting date of the EMA measures varied between people.  During each
assessment, participants responded to the item `How depressed do you feel at
the moment?' on a scale ranging from 0 to 100 to assess their momentary
depression\footnote{Note that this was only one of 27 items that were assessed
  at every measurement occasion. The other items (questions about emotions,
  social expectancies, emotion regulation, context, and psychiatric symptoms)
  are, however, not relevant to this application and, therefore, not further
  described.}.

The initial study procedure was approved by the KU Leuven Social and Societal
Ethics Committee and the KU Leuven Medical Ethics Committee. This secondary
data analysis was approved by the Ethics Review Board of the Tilburg School
of Social and Behavioral Sciences (TSB RP FT16).

\subsection{Analysis and Results}

To correspond to how the GAMs were introduced in this atricle, we applied them
to explore the idiographic processes underlying the data of each individual.
Each GAM was fit using the same specifications as in the simulation, with a
single smooth term for time. Inspecting the estimated processes revealed a
clear picture of the heterogeneity in the processes underlying the data. On one
hand, Figure~\ref{fig:dem_smooth}~(a) shows the ten estimated processes with
the lowest EDF, which appear to be essentially linear. Indeed, for 72 out of
117 participants, the process inferred by the idiographic GAMs is effectively a
linear trend ($EDF < 1.001$). For some of these participants, the linear
estimates appear to be due to strong floor effects where participants
repeatedly indicate depression scores close to zero. However, this explanation
does not hold for all participants, as some participants also received linear
estimates without displaying floor effects. Interestingly, the linear estimates
also indicate an effective absence of dynamic errors, which would be expected
to result in deviations from the linear trend, for these participants

Figure~\ref{fig:dem_smooth}~(b) shows the ten participants with the wiggliest
estimated processes, indicated by the largest EDF\@. For these participants,
the estimated processes clearly deviates from a linear trend, which may either
be due to the presence of non-linearity or the presence of stronger dynamic
errors for these participants. Unfortunately, it is not possible to distinguish
which of these possibilities is correct with this analysis. Additionally,
visual inspection of the non-linearly estimated processes, did not reveal any
common behaviours across the processes of different participants. Nevertheless,
the idiographic processes still displayed interesting behaviours. With respect
to the processes introduced in this article, two participants show particularly
interesting trajectories. First, Figure~\ref{fig:dem_smooth}~(c) shows an
estimated process that closely resembles the decreasing osciallations exhibited
by a damped oscillator. Second, the estimated process of another participant
illustrated in Figure~\ref{fig:dem_smooth}~(d) could be seen as switching
between a state low and a state of high depression, with considerable
osciallations within each state.

\begin{figure}[!t]
  \caption{GAM smooths}
  \fitfigure{demonstration_smooths.png}
  \label{fig:dem_smooth}
\end{figure}

\subsection{\textcolor{red}{Conclusion}}

\textit{WIP}

\section{Discussion}

This results presented in this article have several important limitations.
Since, only a limited selection of processes was used it is possible that the
presented results may not generalize to other processes. However, the used
processes already constitute violations to the smoothness assumption made by
the LPR, GP, and GAM to which these methods demonstrated some robustness. In
addition to this, as explained earlier, using different configurations for the
LPR, GP, and GAM is likely going to change their performances. Further, we
focused on introducing these methods by inferring time dependent non-linear
processes from univariate, single subject data with independent normally
distributed measurement errors. This is a statistically idealized setting,
which does not address many of the goals and challenges researchers are facing
when working with ILD in practice. Frequently, ILD contains measurements for
many individuals on several psychological constructs. Classically, this enables
researchers to study how these constructs vary and interact over time and how
these dynamics differ between people. In addition to that, each construct is
frequently measured using multiple indicators with (ordinal) measurement
errors, which need to be modelled using different psychometric models (e.g.,
factor models, item response models).

There are fortunately many ways in which the presented methods could be adapted
to these more complex data structures. The GP and GAM can be naturally adapted
to incorporate multilevel data without substantially extending the statistical
theory underlying both methods. For the GAM this extension is already
implemented in some software. Another approach may be to study between person
differences in the latent processes using functional data analysis. In this
analysis the individual latent processes are first estimated using one of the
presented data driven techniques. Subsequently, the inferred processes are
treated as function valued data, which can be analyzed to find for example
group differences in a functional ANOVA \parencite{kaufman_bayesian_2010} or to
find the functions which account for the maximum between person variation in a
functional principal component analysis \parencite{aue_prediction_2015}.

Similarly, the GP and by extension the GAMs can model latent variables
underlying for example a factor model, which naturally extends these methods to
a setting with multiple indicator variables. This raises the possibility of
extending these models to typical psychometric measurement models, to
accurately capture more complex measurement error distributions. However, it
is unclear how well these extensions work in practice, beyond the Gaussian
process factor models, which are already implemented in software
\parencite{clark_dynamic_2023}. The parametric models introduced already
include a factor model for the observed variables, which can incorporate
multiple indicators, which may also be non-normally distributed.

Lastly, there are many ways in which the presented methods can be used to study
multivariate data. This is because, even though all methods were used to infer
time dependent non-linear processes in this paper, they can in theory be used
to estimate many smooth and continuous functions (and even non-smooth and
non-continuous functions to the degree presented in this paper). This makes it
possible to for example infer non-linear cross- and autoregressive
relationships from data in discrete time \parencite{wood_generalized_2006,
  rasmussen_gaussian_2006, eleftheriadis_identification_2017} and non-linear
differential equation models in continuous time
\parencite{yildiz_learning_2018}. For the presented processes in particular
this should even be more appropriate, since they were generated using
autoregressive relations. These models also present the exciting possibility to
combine partial parametric models with non-linear data driven functions,
estimated by the presented methods. Lastly, the presented methods could be used
to infer unobserved input variables to parametric dynamic models
\parencite{alvarez_latent_2009,nayek_gaussian_2019}, which could for example
take the form of seasonal and cyclic influences to indicator variables
\parencite{clark_dynamic_2023}. While these extensions and possibilities exist
in theory and are in isolation implemented in software, most are currently not
available and implemented in a form which would allow applied researchers to
flexibly adapt these methods to the characteristics of specific ILD\@.
Therefore it is important that future research, combines and extends the
presented methods and implements these extensions in software that makes them
accessible for applied modelling.

\printbibliography[]

\end{document}