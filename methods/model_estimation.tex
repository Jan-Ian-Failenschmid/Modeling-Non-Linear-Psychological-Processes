\subsection{Model estimation}

\subsubsection{General additive models}

After simulating the deta, four methods were used to infer the state process for 
each data set. First, general additive models (GAM) were fit to each data set 
using the mgcv package. Specifically, a model with a single smooth function 
relating the state value to time was chosen and approximated using $50$ 
thin-plate splines. The weight of the penalty term of the GAM was determined 
using the generalized cross-validation procedure that is incorporated in mgcv. 
The fitted model was then used to obtain state estiamtes at each time point as 
well as respective 95\% Bayesian credible intervals 

\subsubsection{Local polynomial regression}

Secondly, a local polynomial regression was fit to each data set using the 
nprobust package. We used first order local polynomials, resulting in a local 
linear regression. Further, we used a gaussian kernel and optimized the bandwidth 
of the kernel function using the build-in mean squared error optimization. Afterwards, 
bias corrected state estimates and robust standard error were obtained, to 
account for the inherrent bias of the local polynomial model. Lastly, we used 
these estiamtes and standard errors to construct 95\% confidence bands for the
state function. 

\subsubsection{Gaussian processes}

In order to fit the Gaussian processes efficiently for a large number of data 
sets we used a Hilbert space approximate Gaussian process model in Stan.
This approximation uses a linear combination of basis functions that are derived 
from the covariance matrix of the Gaussian process. The number of basis 
function was chosen in accordance to the length scale of the 
Gaussian process and boundary factor of the approximation. This was achieved 
by following the selection procedure detailed by Riutort-Mayol et al. (2023) 
To concur with how Gaussian 
processes are usually applied we selected a zero mean function together with a 
quadratic exponential covariance functions. This covariance function is both symmetric and 
stationary. Point estimates for the state value as well as 95\% credible interval 
were obtained from posterior distribution of the Gaussian process. 

\subsubsection{State-space models}

Lastly, a parametric dynamic model is fit to each data set using the dynr package.
Here, the specific models were adapted to corrspond exactly to the data 
generation models of each respective data set. This mimics the situation in which 
prior theory about the parametric form of the dynamic model exists. Further, each 
model was estimated 10 times using different starting values to minimize the 
risk of obtaining local minimum solutions. Estimates for the smoothed state 
values and the smoothing error covariance were obtained directly and used to 
generate confidence intervals for the state inference.