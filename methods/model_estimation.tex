% -----------------------------------------------------------------------------%
% Title:                                                                       %
% Author: Jan Ian Failenschmid                                                 %
% Created Date: 13-03-2024                                                     %
% -----                                                                        %
% Last Modified: 02-04-2024                                                    %
% Modified By: Jan Ian Failenschmid                                            %
% -----                                                                        %
% Copyright (c) 2024 by Jan Ian Failenschmid                                   %
% E-mail: J.I.Failenschmid@tilburguniveristy.edu                               %
% -----                                                                        %
% License: GNU General Public License v3.0 or later                            %
% License URL: https://www.gnu.org/licenses/gpl-3.0-standalone.html            %
% -----------------------------------------------------------------------------%

\subsection{Model estimation}

After simulating the data sets we fit each of the non-parametric models,
specifically a generalized additive model \parencite{wood_generalized_2006},
a local polynomial regression \parencite{fan_local_1997}, and
a Gaussian process \parencite{rasmussen_gaussian_2006},
as well as a parametric state-space model \parencite{durbin_time_2012} to each
of the data sets. For a detailed description of the model fitting procedure,
see \textbf{Appendix A}. Importantly, the model fitting procedure for each of
the models was validated on a subsample of \textbf{5} data sets per condition,
in order to ensure good model fit. If a model did not
converge for a given data set, the respective data set was removed from the
simulation and another data set was simulated instead. Lastly, it is important
to note that the functional form of the state-space model in each condition
was adjusted to the data generating model, while the same
non-parametric models were used across conditions. After each of the
models was fit to the data, they were used to obtain point and interval
estimates (i.e., 95\% confidence or credible intervals) for the
latent process value at each time point.