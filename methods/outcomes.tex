\subsection{Outcome measures}

\subsubsection{Root mean squared error}

To compare the in-sample predictive accuracy of the state inference accross 
analysis methods and simulagtion conditions, we calculated the root mean 
squared error (RMSE) between the smoothed state inference and the simulated 
state values. The resulting RMSE values were then anaylzed using a repeated measures 
ANOVA, as each data set was analyzed with every method. Thus, the analysis 
method constitues a repeated-measuers effect, wheras the simulation conditions 
were entered into the model as fixed effects. To find evidence of both 
present and absent effects an exhaustative model selection based on AIC and BIC 
model weights was conducted. 

\subsubsection{Generalized cross-validation}

To evaluate the out of sample interpolation predictive performance of each method 
we obtained generalized cross-validation scores (GCV). The GCV 
criterion is a computationally efficient and rotation invariant alternative 
to the ordinary cross-validation criterion that would be obtaiend from a
leave-one-out cross-validation. Because the GAMs, local polynomail regression, 
and GPs are linear smoothers, the GCV can in these cases be obtained directly 
from the influence matrix of the model. In addition, for linear parametric state-space models 
there are approximate analytic results for ontaining the GCV from the innovation 
covariance matrix. However, since in this simulation non-linear and regime 
switching state-space models will be conidered, the GCV is instead calculated 
by performing a leave-one-out cross validation and adjust the resulting errors. 
The GCV values obtained for each analysis methods were anlyzed using the same 
repeated measures ANOVA based model selection procedure as was used for the RMSEs. 

\subsubsection{Confidence interval coverage}

The last performance measure that was inestigated during the simulation 
is the confidence interval coverage probability. That is, the probability that 
the confidence generated by any of the analysis methods for the state inference, 
falls around the true simulated state value. To analyze this, confidence or Bayesian 
credible intervals were obtained for each of the state inferences and it was 
recorded whether the the confidence interval included the true state value or not. 
Afterwards, a multilevel logistic regression was used to predict the probability 
of a confidence interval being around the true state value using the simulation 
conditions and analysis methods as predictors. We again performed a exhaustative
model search based on AIC and BIC model weights to evaluate the evidence for both 
present and absent effects. 