% -----------------------------------------------------------------------------%
% Title:                                                                       %
% Author: Jan Ian Failenschmid                                                 %
% Created Date: 13-03-2024                                                     %
% -----                                                                        %
% Last Modified: 02-04-2024                                                    %
% Modified By: Jan Ian Failenschmid                                            %
% -----                                                                        %
% Copyright (c) 2024 by Jan Ian Failenschmid                                   %
% E-mail: J.I.Failenschmid@tilburguniveristy.edu                               %
% -----                                                                        %
% License: GNU General Public License v3.0 or later                            %
% License URL: https://www.gnu.org/licenses/gpl-3.0-standalone.html            %
% -----------------------------------------------------------------------------%

\subsubsection{Problem}

We conducted a simulation study to evaluate how well each of the introduced
parametric and non-parametric statistical methods recovers different
non-linear processes, which researchers are likely to encounter in ESM
research. To facilitate an accessible introduction to these statistical methods
and apply them under conditions for which software implementations are
currently available, we employed a univariate single-subject design, such that
the simulated data corresponds to repeated measurements of one variable for one
individual. The theory underlying the non-linear analysis methods
(i.e., generalized additive models, local polynomial regression,
Gaussian processes) suggests, that each method in its default configuration
should infer processes most accurately, if they
(a) are without sudden jumps (i.e., continuous),
(b) have constant wigglyness (i.e., constant second derivative), and
(c) if they are smooth (i.e., differentiable).
Thus, we specifically considered processes in this simulation that varied with
regards to these characteristics. Additionally, it is expected that
the inference is more accurate for larger sample sizes. Here we varied both
(d) the overall length of the sampling period and
(e) the sampling frequency, as both of these factors were expected to affect
the inference differently for different processes and methods.

\subsubsection{Design}

To investigate their impact on the performance of the different analysis
methods
we manipulated the continuity, wigglyness, and smoothness, of the underlying
process. To test (a) the continuity and (b) wigglyness we chose four exemplar
processes that vary on these two characteristics. Each of these exemplar
processes represents a theoretically feasible and empirically demonstrated
psychological process which might be interesting to study using ESM.
Depictions of each of the processes can be seen in Figure
\ref{exemplar_no_process_noise}, in which the points represent the noisy
observations and the line shows the latent process value at each time point.

\begin{figure*}
    \caption{Non-linear exemplar processes}
    \fitfigure{figures/exemplar_no_process_noise.png}
    \label{exemplar_no_process_noise}
\end{figure*}

The first two processes depict common growth curves in which the latent
construct approaches a fixed value over time.
Here, the exponential (1) and logistic (2) growth mainly differ in their
respective growth velocity during the start of the process. Growth curves
like these have been found to appear in cognitive and intellectual development
\parencite{mcardle_comparative_2002,kunnen_dynamic_2012}, motor learning
\parencite{newell_time_2001}, and second language acquisition
\parencite{de_bot_dynamic_2007}. Both of these growth curves are continuous and
change their wigglyness over time.

The third process has been generated from a cusp catastrophe model.
This dynamic model exhibits natural jumps or discontinuities when
varying one of its parameters over time. This has made it popular for
describing sudden changes in attitudes \parencite{van_der_maas_sudden_2003},
flow \parencite{ceja_dynamics_2009}, or alcohol use relapse
\parencite{witkiewitz_modeling_2007}. In addition to that, the cusp model also
describes the average behavior of dichotomous network models
\parencite{finnemann_theoretical_2021}, which are a popular way of modelling
clinical symptoms \parencite{borsboom_network_2013}. This makes the cusp an
attractive candidate for modelling therapy effects under these network
theories.

The last depicted process is a damped oscillator. As such it describes the
return to baseline after a perturbation, while oscillating around this
baseline. Because of this behavior, the damped oscillator has been proposed as
a model for affect regulation \parencite{chow_emotion_2005, waugh_affect_2021}.
This model shows a clear decrease in wigglyness over time.
All of the considered processes were generated from state-space models,
such that the value of the process at a given time point is determined by a
(non-) linear function of the previous process value.

To manipulate the (c) smoothness of the process we additionally added a dynamic
or process noise component to the respective state-space models. This means,
that at each time point the process value does not only depend on the previous
process value but also on a random error, that is usually attributed to
external shocks or perturbations. This results in processes that are not smooth
(i.e., non-differentiable), where the degree of the roughness depends on the
variance of the dynamic error component. The dynamic error standard deviations
where chosen in relation to the range of the process. Here error standard
deviations of \textbf{0.25} and \textbf{0.5} were considered reasonable.
Figure \ref{exemplar_process_noise} shows possible realizations of
the exemplar processes with dynamic noise. Importantly, a condition lacking
dynamic noise was intentionally omitted from our study, as dynamic
noise is reasonably expected to be present in all ESM measures.

\begin{figure*}
    \caption{Non-linear exemplar processes}
    \fitfigure{figures/exemplar_process_noise.png}
    \label{exemplar_process_noise}
\end{figure*}

Additionally, we altered the sample size during the simulation. Here
we manipulated either (d) the sampling period or (e) the sampling frequency,
since these are distinct methodological choices that are expected to affect
the performance of the analysis methods differently. For example, since the
local polynomial estimator \parencite{fan_local_1997} only uses data in the
local neighborhood,
extending the sampling period outside the neighborhood might not improve the
estimation. On the contrary, if the process changes its optimal bandwidth over
time, extending the sampling period might even negatively affect the inference.
However, increasing the sampling frequency would provide more data within the
local neighborhood and hence improve the inference. If one instead considers
a method that uses all available data and does not assume constant
wigglyness, such as the generalized additive models, both ways of increasing
the sample size should facilitate the inference similarly.

Therefore, the simulation consisted of four factors that were crossed in a
full factorial design:

\begin{APAenumerate}
    \item Exemplar processes (combining continuity and wigglyness)
    \item Smoothness ($\sigma$ = 0.25; $\sigma$ = 0.5;)
    \item Sample size: Frequency (20 - 225; \textcite{wrzus_ecological_2023})
    \item Sample size: Period length (20 - 225;
    \textcite{wrzus_ecological_2023})
\end{APAenumerate}