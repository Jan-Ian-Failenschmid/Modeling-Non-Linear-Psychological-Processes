A simulation study was conducted, to evaluate the performance of the different 
introduced smoothing techniques in recovering non-linear state processes. 
Specifically, we were interested in the extend to which different types 
of non-linear state functions could be estimated under conditions that are 
realisitcally found in ESM research. For the present simulation we further 
limited considerations to univariate single subject desgins, for simplicity and 
to use each method in the form in which they are currently available in software.
Thus, we manipulated two 
conditions, which were theorized to influence the state recovey. First, the type 
of non-linearity present in the state process, the quality of information 
about the state process.

Based on theory underlying the considered smoothing techniques, we expect that 
all considered smoothing techniques will perform best if the non-linear state process 
is relatively smooth and follows a constant dynamic without regime switching. 
In order to test this, we operationalized smoothness as either low or high 
process noise. Since, we expect there will always be some process noise present 
in ESM data, we did not include a completely smooth condition without process noise. 
Further, we tested two types of regime switching. The first regime switching 
condition changes the dynamic of the simulated process, but not the attractor, 
whereas the second condition, switches the attractor of the state process, but not the 
dynamic. 

Further, we expected that all considered methods will infer the state process 
with higher accuracy the better the quality of information is. To test this, 
we varied two factors. The first is to the signal-to-noise ratio of the 
data, which corresponds to the ratio of the variance of the centered process values 
and the variance of the measurement noise. Thus a larger signal-to-noise ratio 
corresplonds to more variation in the process itself relativ to the measurement 
uncertainty. The second factor corresponds to the sample size or the 
number of observations that are taken over a given time interval. 

Therefore, the simulation consisted of five factors that were crossed in a complete 
factorial desgin: 

\begin{enumerate}
    \item Process noise (small vs. large)
    \item Dynamic regime switching (absent vs. present)
    \item Attractor regime switching (absent vs. present)
    \item Signal-to-noise ratio (small vs. large)
    \item Sample size (small vs. large)
\end{enumerate}