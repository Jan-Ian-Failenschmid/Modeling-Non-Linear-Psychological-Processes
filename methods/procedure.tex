\subsection{Procedure}

In all conditions, the data was generated from a univariate non-linear 
state space model, with a state function described by:

\begin{align*}
    Y_{t+1} &= f_j (Y_{t}) + \eta_{t+1} \\
    \eta_{t+1} &\sim N(0, \sigma^2_{Process})
\end{align*}

Where $f_j$ is either a single continous non-constant function in the single 
regime condition or switching between two different functions according to a 
Markov chain in the regime switching conditions. In the regime swtiching conditions 
these functions were selected in such a way that either the dynamic of the system, 
the attractor, or both change. In addition to that, the variance of the $\eta_{t+1}$
was set according to the process noise condition and the time-points 
were determined by subdevidign the interval from zero to one by the desired sample 
size. After simulating the state values, we centered the state values and calculated
the measurement error that would result in the required signal-to-noise ratio. 
Laslty, the measurement error was added to the simulated state values to generate 
observations. 

$N$ number of data sets were generated per cell of the factorial desgin. This number 
was determined by first running a small pilot simulation with 30 data sets per cell. 
Afterwards, a seperate simulation was conducted on the outcome models. To simulate 
from the outcome models we used the significant parameter estimates from the 
pilot simulation and set the insignificant parameters to zero. Based on these 
preliminary effect sizes $N$ observations per cell resulted in selecting the 
correct outcome model $95\%$ of the time.

