% -----------------------------------------------------------------------------%
% Title:                                                                       %
% Author: Jan Ian Failenschmid                                                 %
% Created Date: 13-03-2024                                                     %
% -----                                                                        %
% Last Modified: 02-04-2024                                                    %
% Modified By: Jan Ian Failenschmid                                            %
% -----                                                                        %
% Copyright (c) 2024 by Jan Ian Failenschmid                                   %
% E-mail: J.I.Failenschmid@tilburguniveristy.edu                               %
% -----                                                                        %
% License: GNU General Public License v3.0 or later                            %
% License URL: https://www.gnu.org/licenses/gpl-3.0-standalone.html            %
% -----------------------------------------------------------------------------%

\subsection{Procedure}

For simulating the data, each of the exemplar processes was represented as a
generative state-space model, such that at each time-point the value of the
process corresponds to a (non-) linear function of the previous value and the
dynamic noise component. In each condition we simulated data from each of the
exemplar processes with the respective dynamic error variance, sampling
frequency, and sampling period length. The parameters of the exemplar processes
were chosen, so that they vary over a comparable range. Lastly, measurement
errors from a standard normal distribution were added to the process values at
each time point to generate noisy observations.
For a detailed and technical explanation of the data
generation process, please see \textbf{Appendix A}.

In order to determine the required amount of data sets per condition,
we conducted a power simulation based on a pilot
sample of 30 generated data sets per cell. Based on these data sets, we
calculated the outcome measures (i.e., one MSE and one GCV value per data set)
to which we fit the appropriate MANOVA model. This MANOVA model was used to
obtain initial effect size estimates for the power simulation, by retaining all
significant parameter estimates and setting all insignificant parameters to
zero. After simulating from this MANOVA model, we employed an exhaustive
AIC and BIC weight based model search, to recover the correct MANOVA
model from the data. In addition to finding evidence in favor of the
effects that are likely present this approach can
also find evidence to indicate which effects are likely to be absent in the
data \parencite{wagenmakers_aic_2004}. Here, \textbf{N} observations per
condition were required to recover the correct model with at least
\textbf{95\%} probability.